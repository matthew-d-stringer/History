\documentclass[12pt]{article} %was 12pt
\usepackage{tikz}
\usetikzlibrary{matrix,arrows}
\usepackage{graphicx}
\usepackage{amsmath} 
\usepackage{amssymb} 
\usepackage{amsthm}
\usepackage{enumitem} % to change appearance of enum and item environments
\usepackage{framed}
\usepackage{color}
\usepackage{multicol}
%Matthew
%\usepackage[utf8]{inputenc}
%\usepackage[english]{babel}
%end Matthew

%\usepackage{yhmath}
% adjust the margins using the geometry package
\usepackage[left=0.75in, right=0.75in, top=0.75in, bottom=0.75in]{geometry}
\usepackage[parfill]{parskip}
 %\usepackage{mathpazo}
%\usepackage{euler}


% customize the headers using the fancyhdr package
\usepackage{fancyhdr}
\pagestyle{fancy}

\usepackage{hyperref}

\usepackage{mathrsfs}
\usepackage{mathtools}

\renewcommand{\headrulewidth}{0.4pt}
\renewcommand{\footrulewidth}{0.4pt}

\newenvironment{hwproblem}[1]{{\large \bfseries Problem #1\,:} \begin{trivlist}\item[]\vspace{-0.5ex}}{\end{trivlist}\vspace{3ex}}
\newenvironment{hwproblemquestion}[1]{{\bfseries Question #1\,:} \begin{trivlist}\item[]\vspace{-1.5ex}}{\end{trivlist}\vspace{3ex}}


% customize enumerate and itemize environments
\setlist[itemize]{labelsep=1ex,itemsep=1.5ex,parsep=0ex,leftmargin=4ex,topsep=0.5ex}
\setlist[enumerate]{labelsep=1ex,itemsep=1.5ex,parsep=0ex,leftmargin=4ex,topsep=0.5ex}




\rhead{Honors US History}
\lhead{Matthew Stringer} % your name
\lfoot{}
\cfoot{Notes}  % change to the corresponding number
\rfoot{\thepage}

\setlength{\headheight}{15pt} 
\title{Chapter 21 Notes} % change to the corresponding number
%\date{September 5, 2018}
\author{Matthew Stringer} % your name and ID number

% Anything above the \begin{document} is the template. If you wish to start a new document using this template, erase everything inside of the \begin{document}...\end{document}
\allowdisplaybreaks
\setlength{\parindent}{4em}
\begin{document}
\maketitle
\tableofcontents
\newpage

\section{Changes in the early 20th Century and the Origins of Progressivism}
\subsection{Rapid industrialization, immigration and urbanization in the late 1800s had changed America greatly}
\begin{enumerate}
	\item The Progressives were diverse. They included:
	\begin{itemize}
		\item Protestant Church leaders
		\item African Americans
		\item Union Leaders
		\item Feminists
	\end{itemize}
	\item They believed that society badly needed changes and that government was the proper agency
		for correcting social and economic ills.
	\item Muckrakers were writers that attempted to expose corporations and wrote in depth 
		investigative stories. They exposed inequalities, educated the public about corruption in
		high places, and prepared the way for corrective action.
\end{enumerate}

\subsection{Progressivism is based on faith in democracy.}
\begin{enumerate}
	\item Voter Participation:
	\begin{itemize}
		\item In 1910, states adopted a system for voters to use a state issued ballot and voters 
			mark their choices secretly in a private booth. This was first done by Massachusetts in
			1888. 
		\item Primaries try nominate politicians more democratically. Before they were decided by 
			party bosses. Primaries weren't very effective because party bosses implemented a 
			confusing system so they could maintain control.
		\item Senators used to be chosen by state legislatures. In 1899 Nevada was the firs state to
			give voters direct vote of their senators. By 1913, the 17th Amendment required all 
			senators be chosen by popular vote.
		\item Amendments to the state legislatures offered voters:
		\begin{enumerate}
			\item the \textit{initiative}: a method by which voters could compel the legislature to
				consider a bill
			\item the \textit{referendum}: a method that allowed citizens to vote on proposed laws 
				printed on their ballots.
			\item the \textit{recall}: a method for voters to remove a corrupt or unsatisfactory
				politician from office by majority vote before the official's term expired.
		\end{enumerate}
	\end{itemize}
	\item Reforms at the city level:
	\begin{itemize}
		\item City bosses and their corrupt alliance with local businesses were the first target by
			progressives.
		\item Mayor Samuel M. "Golden Rule" Jones introduced a comprehensive program that created 
			free kindergartens, night schools, and public playgrounds.
		\item Reform leaders wanted to take control of Public Utilities.
		\item By 1915, two-thirds of the nation's cities owned their own water systems.
		\item Progressives also wanted to be able to elect the heads of city departments like fire,
			police, and sanitation.
		\item By 1923, more than 300 cities had adopted this policy(manager-council plan)
	\end{itemize}
	\item State Reforms:
	\begin{itemize}
		\item Shutting down saloons and prohibit alcohol was an issue that the reformists were 
			divided.
		\item Progressives believed that saloons were the headquarters of political machines, they
			did not believe in the temperance movement.
		\item By 1915, prohibitionists persuaded the legislatures of two-thirds of the states to 
			prohibit the sale of alcohol.
		\item Progressives lobbied for better schools, juvenile courts, liberalized divorce laws and
			safety regulations for tenements and factories.
		\item Reformers also fought for a system of parole, separate reformations for juveniles, and
			limits on the death penalty.
		\item Progressives fought to keep children out of mines and factories. This was done most 
			effectively with compulsory school attendance laws. 
	\end{itemize}
\end{enumerate}

\section{National Reforms and Progressive Presidents}
\subsection{Progressive Reforms at the federal level (1901-1912)}
\begin{enumerate}
	\item Roosevelt believed that the president's job included setting the legislative agenda for
		Congress
	\item Roosevelt insisted on a \textit{Square Deal} in any labor disputes.
	\item In 1902, Roosevelt mediated a strike that took place in coal mines. His nonpartisan policy
		created a deal which granted the miners a 10 percent wage increase along with a 9-hour work
		day.
	\item Roosevelt increased his popularity by busting many overpowered trusts. He busted more than
		40 large corporations.
	\item He created the Pure Food and Drug Act which forbade the manufacture, sale, and 
		transportation of adulterated or mislabeled food and drugs.
	\item He also created the Meat Inspection Act which required that federal inspectors visit
		meatpacking plants to ensure that they meet minimum standards of sanitation.
\end{enumerate}
\subsection{Lead by Eugene V. Debs, the Socialist platform was more radical than Progressives}
\begin{enumerate}
	\item The Socialist Party wanted public ownership of railroads, utilities, and even of major 
		industries such as oil and steel.
	\item Debs was a former railway union leader and he was an outspoken critic of business and a 
		champion of labor.
	\item Some Socialist ideas were adopted, including: 
	\begin{itemize}
		\item Public ownership of utilities
		\item The eight-hour workday
		\item Pensions for employees
	\end{itemize}
\end{enumerate}

\section{African-Americans in the Progressive Era}
\begin{enumerate}
	\item Progressive presidents only wanted to help the white race.
	\item President Wilson allowed permitted the segregation of federal workers and buildings
	\item Few Progressives did anything about racist segregation and lynching..
	\item At the end of the 19th century, 90\% of African Americans lived in the South
	\item Between 1910 and 1930, millions traveled north because 
	\begin{itemize}
		\item deteriorating race relations
		\item destruction of their cotton crops by boll weevil
		\item job opportunities in the northern factories opened up after World War 1.
	\end{itemize}
	\item In 1905, W. E. B. Du Bois alongside a group of black intellectuals to found the 
		\textit{Niagara Movement}
	\item On Lincoln's birthday in 1908, Du Bois and other members of the Niagara movement founded
		the National Association for the Advancement of Colored People (NAACP)
\end{enumerate}

\section{Women and the Progressive Movement}
\begin{enumerate}
	\item Carrie Chapman Catt became the president of the National American Woman Suffrage 
		Association (NAWSA)
	\item She argued that that the vote would broaden democracy and empower women, and allow them to
		actively care for their family.
	\item Some women took to the streets with pickets, parades, and hunger strikes.
	\item In the 19th Amendment, women's right to vote was guaranteed 
\end{enumerate}

\end{document}
