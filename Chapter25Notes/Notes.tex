\documentclass[12pt]{article} %was 12pt
\usepackage{HistoryTemplate}

\rhead{Honors US History}
\lhead{Matthew Stringer} % your name
\lfoot{}
\cfoot{Notes}  % change to the corresponding number
\rfoot{\thepage}

\setlength{\headheight}{15pt} 
\title{Chapter 25 Notes} % change to the corresponding number
%\date{September 5, 2018}
\author{Matthew Stringer} % your name and ID number

% Anything above the \begin{document} is the template. If you wish to start a new document using this template, erase everything inside of the \begin{document}...\end{document}
\allowdisplaybreaks
\setlength{\parindent}{4em}

\begin{document}
	\maketitle
	\tableofcontents
	\newpage
	\section{Paragraph}
	\textbf{Question:} What events contributed to changes in American foreign policy from isolation to intevention? \newline
	There were several reasons that the United States foreign policy changed, however in order to analyze them,
	we first must look at why we remained neutral. In the early 1930s, many Americans felt that involvement in 
	World War 1 was a big mistake because it only served greedy bankers and arms dealers. Because of this, many 
	Americans were very reluctant to join another World War. Next, there was the Neutrality Acts. These acts limited
	the government's ability to aid foreign entities, including Loyalists in the Spanish Civil War. Although many
	Americans wanted to assist Loyalists in the war, however, the Neutrality Acts forbade this. Finally, there was 
	the America First Commitee, which rallied people around the country to avoid getting involved in Europe's troubles.
	As the war progressed, many Americans were frightened by German forces conquering one country after another, however
	they were still reluctant to supporting or joining the war. When France was taken by Germany, the public agreed that
	it was permissable to increase the defense budget, but giving direct aid to Britain was controversial. As time 
	progressed, Roosevelt slowly became more prepared for involvement in the war, to the point that he initiated 
	a peace-time draft. Over the Pacific, the relationship with Japan was strained. When Japan joined the Axis powers,
	the United States immediately responded with embargo that limmited the trade of essential materials going to Japan,
	including oil. Japan realized that they must attack soon so that they can limit U.S. positioning in the Pacific 
	before they run out of oil. The U.S., however, hoped to delay confrontation so they could prepare. This eventually
	lead to the Attack of Pearl Harbor that 2,400 Americans and sunk 20 warships. This was the final straw for Americans 
	and within a day, lead to war with Japan and the Axis powers.
	\newpage

	\section{American Foreign Policy in 1930s}

	\subsection{Herbert Hoover's Foreign Policy}

	\subsubsection{Japanese Aggression}
	\begin{itemize}
		\item In September 1931, Japan marched into Manchuria and established a puppet government
		\item Although the League of Nations threatened to take action, all they did was sign a
			resolution to condeming Japan. 
		\item Japan finally left the League of Nations and never joined again.
		\item The U.S. responded by refusing to recognize the legitimacy of the new puppet government
		\item This did little to prevent future aggression.
	\end{itemize}

	\subsubsection{Latin America}
	\begin{itemize}
		\item Throughout Hoover's presidency, he maintained a healthy relationship with Latin America.
		\item He pulled troops to leave Nicaragua and negotiated a treaty with Haiti.
	\end{itemize}

	\subsection{Franklin Roosevelt's Policies, 1933-1938}
	\subsubsection{Good-Neighbor Policy}
	\begin{itemize}
		\item During Roosevelt's presidency, U.S. Delegates met at the Seventh Pan-American Conference
			in Uruguay where they agreed to never again intervene in the internal affairs of Latin 
			America.
		\item In 1936, another Pan-American Conference was held and Roosevelt personally attended this one.
		\item In this meeting, Roosevelt united the group of several nations in order to protect each other
			in the face of war.
		\item In 1934, FDR persuaded Congress to nullify the Platt Amendment, that forced Cuba's Foreign Policy
			to be approved by the U.S. in exchange for Guantanamo Bay.
	\end{itemize}

	\section{Isolationism and Neutrality- beliefs and policies}

	\subsection{Learning from World War 1}
	\begin{itemize}
		\item In the 1930s, many Americans believed that the U.S. envolvement in World War 1 was a huge mistake.
		\item Because of this, the public felt uneasy about joining World War 2.
	\end{itemize}

	\subsection{Neutrality Acts}
	\begin{itemize}
		\item Isolationist senators and representatives were the majority in 1938 
		\item They had Roosevelt sign documents in order to stay neutral
	\end{itemize}

	\subsection{Spanish Civil War}
	\begin{itemize}
		\item Although many Americans wanted to aid the Loyalists, they couldn't 
			because the Neutrality Acts forbade it.
	\end{itemize}

	\subsection{America First Commitee}
	\begin{itemize}
		\item In 1940, a group of Americans were concerned with FDR's large support for Britain.
		\item They formed the America First Commitee and rallied people from across the nation
			to promote isolationism.
	\end{itemize}

	\section{Events that shifted from neutrality to war}
	\subsection{Appeasement}
	\begin{itemize}
		\item While Germany invaded small countries and violate the Treaty of Versailles, Britain and France
			adopted the policy of `Appeasement' which did little to stop Germany.
	\end{itemize}

	\subsection{Cash and Carry}
	\begin{itemize}
		\item The U.S. ended its arms embargo in order to support Britain with weaponry.
		\item If a citizen wanted to transport American weapons to another country, they had to pay in cash
			and carry it themselves.
		\item Although, in theory, this policy was neutral, it only helped Britain since they still controlled
			the ocean.
	\end{itemize}

	\subsection{Selective Service Act(1940)}
	\begin{itemize}
		\item Roosevelt pushed neutrality back one step by enacting mandatory military service.
		\item Isolationists opposed a peacetime draft however they were outnumbered by the majority who 
	\end{itemize}

	\section{Actions in Japan}

	% \section{Home front- impact of the war on civilians}

	% \section{Battlefield- conditions, turning points}

	% \section{Ending the war}

	\newpage
\end{document}
