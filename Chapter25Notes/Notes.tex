\documentclass[12pt]{article} %was 12pt
\usepackage{HistoryTemplate}

\rhead{Honors US History}
\lhead{Matthew Stringer} % your name
\lfoot{}
\cfoot{Notes}  % change to the corresponding number
\rfoot{\thepage}

\setlength{\headheight}{15pt} 
\title{Chapter 25 Notes} % change to the corresponding number
%\date{September 5, 2018}
\author{Matthew Stringer} % your name and ID number

% Anything above the \begin{document} is the template. If you wish to start a new document using this template, erase everything inside of the \begin{document}...\end{document}
\allowdisplaybreaks
\setlength{\parindent}{4em}

\begin{document}
	\maketitle
	\tableofcontents
	\newpage
	\section{Paragraph}
	\textbf{Question:} What events contributed to changes in American foreign policy from isolation to intevention? \newline
	There were several reasons that the United States foreign policy changed, however in order to analyze them,
	we first must look at why we remained neutral. In the early 1930s, many Americans felt that involvement in 
	World War 1 was a big mistake because it only served greedy bankers and arms dealers. Because of this, many 
	Americans were very reluctant to join another World War. Next, there was the Neutrality Acts. These acts limited
	the government's ability to aid foreign entities, including Loyalists in the Spanish Civil War. Although many
	Americans wanted to assist Loyalists in the war, however, the Neutrality Acts forbade this. Finally, there was 
	the America First Commitee, which rallied people around the country to avoid getting involved in Europe's troubles.
	As the war progressed, many Americans were frightened by German forces conquering one country after another, however
	they were still reluctant to supporting or joining the war. When France was taken by Germany, the public agreed that
	it was permissable to increase the defense budget, but giving direct aid to Britain was controversial. As time 
	progressed, Roosevelt slowly became more prepared for involvement in the war, to the point that he initiated 
	a peace-time draft. Over the Pacific, the relationship with Japan was strained. When Japan joined the Axis powers,
	the United States immediately responded with embargo that limmited the trade of essential materials going to Japan,
	including oil. Japan realized that they must attack soon so that they can limit U.S. positioning in the Pacific 
	before they run out of oil. The U.S., however, hoped to delay confrontation so they could prepare. This eventually
	lead to the Attack of Pearl Harbor that 2,400 Americans and sunk 20 warships. This was the final straw for Americans 
	and within a day, lead to war with Japan and the Axis powers.
	\newpage

	\section{American Foreign Policy in 1930s}

	\subsection{Herbert Hoover's Foreign Policy}

	\subsubsection{Japanese Aggression}
	\begin{itemize}
		\item In September 1931, Japan marched into Manchuria and established a puppet government
		\item Although the League of Nations threatened to take action, all they did was sign a
			resolution to condeming Japan. 
		\item Japan finally left the League of Nations and never joined again.
		\item The U.S. responded by refusing to recognize the legitimacy of the new puppet government
		\item This did little to prevent future aggression.
	\end{itemize}

	\subsubsection{Latin America}
	\begin{itemize}
		\item Throughout Hoover's presidency, he maintained a healthy relationship with Latin America.
		\item He pulled troops to leave Nicaragua and negotiated a treaty with Haiti.
	\end{itemize}

	\subsection{Franklin Roosevelt's Policies, 1933-1938}
	\subsubsection{Good-Neighbor Policy}
	\begin{itemize}
		\item During Roosevelt's presidency, U.S. Delegates met at the Seventh Pan-American Conference
			in Uruguay where they agreed to never again intervene in the internal affairs of Latin 
			America.
		\item In 1936, another Pan-American Conference was held and Roosevelt personally attended this one.
		\item In this meeting, Roosevelt united the group of several nations in order to protect each other
			in the face of war.
		\item In 1934, FDR persuaded Congress to nullify the Platt Amendment, that forced Cuba's Foreign Policy
			to be approved by the U.S. in exchange for Guantanamo Bay.
	\end{itemize}

	\section{Isolationism and Neutrality- beliefs and policies}

	\subsection{Learning from World War 1}
	\begin{itemize}
		\item In the 1930s, many Americans believed that the U.S. envolvement in World War 1 was a huge mistake.
		\item Because of this, the public felt uneasy about joining World War 2.
	\end{itemize}

	\subsection{Neutrality Acts}
	\begin{itemize}
		\item Isolationist senators and representatives were the majority in 1938 
		\item They had Roosevelt sign documents in order to stay neutral
	\end{itemize}

	\subsection{Spanish Civil War}
	\begin{itemize}
		\item Although many Americans wanted to aid the Loyalists, they couldn't 
			because the Neutrality Acts forbade it.
	\end{itemize}

	\subsection{America First Commitee}
	\begin{itemize}
		\item In 1940, a group of Americans were concerned with FDR's large support for Britain.
		\item They formed the America First Commitee and rallied people from across the nation
			to promote isolationism.
	\end{itemize}

	\section{Events that shifted from neutrality to war}
	\subsection{Appeasement}
	\begin{itemize}
		\item While Germany invaded small countries and violate the Treaty of Versailles, Britain and France
			adopted the policy of `Appeasement' which did little to stop Germany.
	\end{itemize}

	\subsection{Cash and Carry}
	\begin{itemize}
		\item The U.S. ended its arms embargo in order to support Britain with weaponry.
		\item If a citizen wanted to transport American weapons to another country, they had to pay in cash
			and carry it themselves.
		\item Although, in theory, this policy was neutral, it only helped Britain since they still controlled
			the ocean.
	\end{itemize}

	\subsection{Selective Service Act(1940)}
	\begin{itemize}
		\item Roosevelt pushed neutrality back one step by enacting mandatory military service.
		\item Isolationists opposed a peacetime draft however they were outnumbered by the majority who 
	\end{itemize}

	\section{Actions in Japan}
	\begin{itemize}
		\item Through 1940 and 1941, U.S. relations with Japan were becoming increasingly strained as a result
			of Japan's invasion of China and its ambitions to conquest Southeast Asia.
		\item By allying with Germany and the Axis Powers, Japan was able to continue its expansion.
	\end{itemize}

	\subsection{U.S. Economic Action}
	\begin{itemize}
		\item Roosevelt prohibited the export of steel and scrap iron to all countries except Britain and 
			and countries in the Western Hemisphere when Japan join the Axis powers.
		\item Japan called this an ``unfriendly act"
		\item When Japan occupied French Indochina in July 1941, the U.S. froze all Japanese credits and 
			cut off Japanese access to vital materials such as U.S. oil.
	\end{itemize}

	\subsection{Negotiations}
	\begin{itemize}
		\item Roosevelt  and Secretary of State Cordell Hull insisted that Japan pull troops out of China.
		\item The Japanese ambassador and the U.S. tried to negotiate a change in U.S. policy regarding 
			oil.
		\item In October, General Hideki Tojo made a final attempt at negotiating an agreement, but no 
			one changed their opinion
		\item The U.S. wanted to delay confrontation in the Pacific so that they could build up an army, 
			however, Japan wanted to engage faster.
	\end{itemize}

	\subsection{Pearl Harbor}
	\begin{itemize}
		\item On December 7, 1941, Japanese planes flew over Pearl Harbor bombing every ship in sight.
		\item After 2 hours, 2,400 Americans were killed, almost 1,200 were wounded, 20 warships were
			sunk or severely damaged, and approximately 150 airplanes were destroyed.
		\item Americans were surprised about the attack, but high ranking officials knew that it was 
			was inevitable because they broke the Japanese codes.
		\item Roosevelt described December 7 as ``a date which will live in infamy" and Congress declared
			war on December 8.
		\item On December 11, Germany and Italy declared war on the U.S.
	\end{itemize}

	\section{Home front- impact of the war on civilians}

	\subsection{Mobilization}
	\begin{itemize}
		\item The U.S. and Allied armed forces depended on mobilizing America's people, industries, and 
			creative and scientific communities.
		\item The U.S. government organized a number of special agencies to mobilize U.S. economic and 
			military resources for the wartime crisis. 
		\item The Office of War Mobilization (OWM) set production priorities and controlled raw materials
		\item The Office of Price Administration regulated every aspect of civilians' lives by freezing
			prices, wages, and rents and rationing such commodities as meat, sugar, gasoline, and auto tires
			to fight wartime inflation.
		\item U.S. industries were booming and exceeded their productions and profits of the 1920s.
		\item In 1944, unemployment had disappeared.
		\item American factories produced over 300,000 planes, 100,000 tanks, and ships with a total 
			capacity of 53 million tons.
		\item The 100 largest corporations accounted for up to 70\% of wartime manufacturing.
		\item Labor unions and large corporations agreed that while the war lasted, there would be no 
			strikes. 
		\item Workers were disgruntled when their wages were frozen while corporations made large profits.
		\item The government was empowered to take over companies threatened by strike by the Smith Conally 
			Anti-Strike Act of 1943.
		\item The government paid for the war by increasing income tax and selling war bonds.
		\item The Office of War Information controlled news about troop movements and battles and made sure 
			that the war was seen positively.
	\end{itemize}

	\subsection{The War's Impact on Society}

	\subsubsection{African Americans}
	\begin{itemize}
		\item 1.5 million African Americans left the South.
		\item A million young men served in the Armed forces.
		\item Both soldiers and civilians faced discrimination.
		\item The Roosevelt Administration issued an executive order to prohibit discrimination in government
			and in businesses that received federal contracts.
	\end{itemize}

	\subsubsection{Mexican Americans}
	\begin{itemize}
		\item Mexian Americans worked in defense industries and over 300,000 served in the military.
		\item A sudden influx of Mexican immigrants into Los Angeles stirred white resentment and 
			led to the so-called zoot suit riots in the summer of 1943.
	\end{itemize}

	\subsubsection{American Indians}
	\begin{itemize}
		\item 25,000 American Indians served in the military and thousands more worked in defense industries.
		\item More than half never returned to their reservations.
	\end{itemize}

	\subsubsection{Japanese Americans}
	\begin{itemize}
		\item Japanese Americans suffered from their association with a wartime enemy.
		\item 20,000 native-born Japanese Americans served loyally in the military, however Japanese were suspected
			to be potential spies and saboteurs.
		\item The U.S. government ordered 100,000 Japanese Americans on the West Coast to reside in interment camps.
	\end{itemize}

	\subsubsection{Women}
	\begin{itemize}
		\item Over 200,000 women served in uniform in the army, navy and marines, but in noncombat roles.
		\item An acute labor shortage caused 5 million women to take jobs vacated by men in uniform.
		\item The song about ``Rosie the Riveter" was used to encourage women to take defense jobs.
	\end{itemize}

	\subsubsection{Wartime Solidarity}
	\begin{itemize}
		\item The New Deal helped immigrant groups feel more included and serving together as ``bands of brothers'' in
			combat or working together for a common cause helped to reduce prejudices based on nationality, ethnicity,
			and religion.
	\end{itemize}

	\section{Battlefield-conditions, turning points}

	\subsection{Fighting Germany}
	\begin{itemize}
		\item British and Americans concentrated on overcoming German submarines and bombing on German cities.
		\item The Allies used radar, sonar, and the bombing of German naval bases in order to counter the might of submarines.
		\item The Allies also had to force Germans out of North Africa.
		\item In November 1942, General Dwight Eisenhower and General Bernard Montgomery took North Africa from the Germans.
		\item On D-Day, British, Canadian, and American forces under General Eisenhower secured several beachheads on 
			Normandy coast.
		\item By August of 1944, Paris was liberated and by September, the Allies began the final push to Berlin.
		\item As U.S. troops advanced through Germany, they came upon German concentration camps.
		\item The world was shocked by the discovery of 6 million killed Jews.
	\end{itemize}

	\subsection{Fighting Japan}
	\begin{itemize}
		\item By decoding Japanese messages, the U.S. was able to destroy 4 Japanese carriers and 300 planes in the 
			battle of Midway.
		\item This ended Japanese expansion.
		\item The U.S. began a campaign hopping from island to island and slowly approached Japan.
		\item On August 6 and 9 of 1944, the U.S. dropped atom bombs on Hiroshima and Nagasaki to prevent a land invasion.
		\item Japan Surrendered a week after the second bomb fell.
	\end{itemize}

	\section{Ending the war}
	\begin{itemize}
		\item In Casablanca, Churchill and Roosevelt agreed to invade Sicily and Intaly and demand unconditional surrender 
			from the Axis.
		\item In Teheran, Churchill, Roosevelt, and Stalin met in the Iranian city and agreed for Britain and America to 
			liberate France in the spring and the Soviets would invade Germany and eventually join the war against Japan.
		\item In February 1945, Stalin, Roosevelt, and Churchill met in Yalta and agreed that 
		\begin{itemize}
			\item Germany would be divided into occupation zones
			\item There would be free elections in the liberated countries of Eastern Europe
			\item The Soviets would enter war with Japan
			\item The Soviets would control the southern half of Sakhalin Island and the Kurile Islands in the Pacific 
				and would have special concessions in Manchuria
			\item A new world peace organization (the U.N.) would be formed at a conference in San Francisco.
		\end{itemize}
	\end{itemize}

	\subsection{Costs}
	\begin{itemize}
		\item 50 million military personnel died worldwide.
		\item 15 million Americans served in uniform and 300,000 Americans died.
		\item 800,000 Americans were injured.
	\end{itemize}

	\newpage
\end{document}
