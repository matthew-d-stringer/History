\documentclass[12pt]{article} %was 12pt
\usepackage{HistoryTemplate}

\rhead{Honors US History}
\lhead{Matthew Stringer} % your name
\lfoot{}
\cfoot{Notes}  % change to the corresponding number
\rfoot{\thepage}

\setlength{\headheight}{15pt} 
\title{Chapter 25 Notes} % change to the corresponding number
%\date{September 5, 2018}
\author{Matthew Stringer} % your name and ID number

% Anything above the \begin{document} is the template. If you wish to start a new document using this template, erase everything inside of the \begin{document}...\end{document}
\allowdisplaybreaks
\setlength{\parindent}{4em}

\begin{document}
	\maketitle
	\tableofcontents
	\newpage

	\section{American Foreign Policy in 1930s}

	\subsection{Herbert Hoover's Foreign Policy}

	\subsubsection{Japanese Aggression}
	\begin{itemize}
		\item In September 1931, Japan marched into Manchuria and established a puppet government
		\item Although the League of Nations threatened to take action, all they did was sign a
			resolution to condeming Japan. 
		\item Japan finally left the League of Nations and never joined again.
		\item The U.S. responded by refusing to recognize the legitimacy of the new puppet government
		\item This did little to prevent future aggression.
	\end{itemize}

	\subsubsection{Latin America}
	\begin{itemize}
		\item Throughout Hoover's presidency, he maintained a healthy relationship with Latin America.
		\item He pulled troops to leave Nicaragua and negotiated a treaty with Haiti.
	\end{itemize}

	\subsection{Franklin Roosevelt's Policies, 1933-1938}
	\subsubsection{Good-Neighbor Policy}
	\begin{itemize}
		\item During Roosevelt's presidency, U.S. Delegates met at the Seventh Pan-American Conference
			in Uruguay where they agreed to never again intervene in the internal affairs of Latin 
			America.
		\item In 1936, another Pan-American Conference was held and Roosevelt personally attended this one.
		\item In this meeting, Roosevelt united the group of several nations in order to protect each other
			in the face of war.
		\item In 1934, FDR persuaded Congress to nullify the Platt Amendment, that forced Cuba's Foreign Policy
			to be approved by the U.S. in exchange for Guantanamo Bay.
	\end{itemize}

	\section{Isolationism and Neutrality- beliefs and policies}

	\subsection{Learning from World War 1}
	\begin{itemize}
		\item In the 1930s, many Americans believed that the U.S. envolvement in World War 1 was a huge mistake.
		\item Because of this, the public felt uneasy about joining World War 2.
	\end{itemize}

	\subsection{Neutrality Acts}
	\begin{itemize}
		\item Isolationist senators and representatives were the majority in 1938 
		\item They had Roosevelt sign documents in order to stay neutral
	\end{itemize}

	\subsection{Spanish Civil War}
	

	% \section{Events that shifted from neutrality to war}

	% \section{Actions in Japan}

	% \section{Home front- impact of the war on civilians}

	% \section{Battlefield- conditions, turning points}

	% \section{Ending the war}

	\newpage
\end{document}
