\documentclass[12pt]{article} %was 12pt
\usepackage{tikz}
\usetikzlibrary{matrix,arrows}
\usepackage{graphicx}
\usepackage{amsmath} 
\usepackage{amssymb} 
\usepackage{amsthm}
\usepackage{enumitem} % to change appearance of enum and item environments
\usepackage{framed}
\usepackage{color}
\usepackage{multicol}
%Matthew
%\usepackage[utf8]{inputenc}
%\usepackage[english]{babel}
%end Matthew

%\usepackage{yhmath}
% adjust the margins using the geometry package
\usepackage[left=0.75in, right=0.75in, top=0.75in, bottom=0.75in]{geometry}
\usepackage[parfill]{parskip}
 %\usepackage{mathpazo}
%\usepackage{euler}


% customize the headers using the fancyhdr package
\usepackage{fancyhdr}
\pagestyle{fancy}

\usepackage{hyperref}

\usepackage{mathrsfs}
\usepackage{mathtools}

\renewcommand{\headrulewidth}{0.4pt}
\renewcommand{\footrulewidth}{0.4pt}

\newenvironment{hwproblem}[1]{{\large \bfseries Problem #1\,:} \begin{trivlist}\item[]\vspace{-0.5ex}}{\end{trivlist}\vspace{3ex}}
\newenvironment{hwproblemquestion}[1]{{\bfseries Question #1\,:} \begin{trivlist}\item[]\vspace{-1.5ex}}{\end{trivlist}\vspace{3ex}}


% customize enumerate and itemize environments
\setlist[itemize]{labelsep=1ex,itemsep=1.5ex,parsep=0ex,leftmargin=4ex,topsep=0.5ex}
\setlist[enumerate]{labelsep=1ex,itemsep=1.5ex,parsep=0ex,leftmargin=4ex,topsep=0.5ex}




\rhead{Honors US History}
\lhead{Matthew Stringer} % your name
\lfoot{}
\cfoot{Notes}  % change to the corresponding number
\rfoot{\thepage}

\setlength{\headheight}{15pt} 
\title{Chapter 22 Notes} % change to the corresponding number
%\date{September 5, 2018}
\author{Matthew Stringer} % your name and ID number

% Anything above the \begin{document} is the template. If you wish to start a new document using this template, erase everything inside of the \begin{document}...\end{document}
\allowdisplaybreaks
\setlength{\parindent}{4em}
\begin{document}
\maketitle
\tableofcontents
\newpage

\section{Politics (Pages 475-477)}
\begin{enumerate}
	\item The Republican party dominated both the White House and Congress.
	\item Republicans did not want laissez-faire economics but rather wanted limited government regulation as an aid to sabilizing business.
		The regulatory commissions established in the Progressive era were now ran by appointees who were more sympathetic to business than
		the general public. The main idea was that the nation would benefit if business and the pursuit of profits took the lead in developing
		the economy.
	\item 
	\begin{itemize}
		\item Warren Harding
		\begin{itemize}
			\item Appointed able men to his cabnit including 
			\begin{itemize}
				\item Presidential Canidate and Supreme Court justice Charles Evans Hughes to be secretary of state
				\item An admired former mining engineer and Food Administration leader Herbert Hoover to be \newline secretary of commerce
				\item Pittsburgh industrialist and millionaire Andrew Mellon to be secretary of the treasury
			\end{itemize}
			\item Domestic Policy
			\begin{itemize}
				\item Reduced income tax
				\item Increasd tariff rates through Fordney-McCumber Tariff Act of 1922
				\item established the Bureau of the Budget.
				\item Pardened Eugene Debs
			\end{itemize}
			\item Scandals and Death
			\begin{itemize}
				\item Appointed incompetent and dishonest men to fill imported position including
				\begin{itemize}
					\item Secretary of Interior Albert B. Fall (Accepted brimes for granting oi leases near Teapot Dome, Wyoming)
					\item Attorney General Harry M. Daugherty (Accepted bribes for agreeing not to prosecute certain criminal suspects)
				\end{itemize}
			\end{itemize}
		\end{itemize}
		\item Calvin Coolidge
		\begin{itemize}
			\item The Election of 1924
			\begin{itemize}
				\item Calvin Coolidge because the Dmocrats voted for a 3rd party canidate
				\item Suprisingly, the 3rd party did very well
			\end{itemize}
			\item Vetos and Inaction
			\begin{itemize}
				\item Calvin Coolidge did little but except keep a close watch on the budget.
				\item Coolidge vetoed even the acts of the Republican majority.
				\item This included bonuses for World War 1 veterans.
				\item This also included an act (the McNary-Haugen Bill of 1928) to help 
					farmers as crop prices fell.
			\end{itemize}
		\end{itemize}
	\end{itemize}
\end{enumerate}

\section{Economics (Pages 477-479)}
\begin{enumerate}
	\item Strengths and Weaknesses of 1920's economy:
	\begin{itemize}
		\item Strengths
		\begin{itemize}
			\item Unemployment was below 4\%
			\item The standard of living for Americans improved significantly
			\item Indoor plumbing and central heating became commonplace
			\item Two-thirds of all homes had electricity
		\end{itemize}
		\item Weaknesses
		\begin{itemize}
			\item Prosperity was not universal
			\item 40\% of families were in the poverty range, struggling to live off of \$1,500 a year
			\item Farmers did not participate in the booming economy
		\end{itemize}
	\end{itemize}
	\item The most significant cause of business prosperity was increased productivity because it 
		allowed the mass production of many new consumer goods of the 1920's. Without the assembly line,
		many new goods would take too long to manufacture in order to make a profit.
	\item The automobile changed life in 
	\begin{itemize}
		\item Social/Cultural changes
		\begin{itemize}
			\item Affected: 
			\begin{itemize}
				\item Traveling for pleasure
				\item Commuting to work
				\item dating
			\end{itemize}
			\item Created new problems including:
			\begin{itemize}
				\item Traffic jams
				\item Injuries and deaths due to accidents
			\end{itemize}
		\end{itemize}
		\item Economic Impact
		\begin{itemize}
			\item Other industries such as steel, glass, rubber, gasoline, and highway construction
				depended on automobile sales
		\end{itemize}
		\item Political Effects
		\begin{itemize}
			\item 
		\end{itemize}
	\end{itemize}
\end{enumerate}

\section{Culture and Society (Pages 479-486)}

\section{Values in Conflict}

\section{Ku Klux Klan}

\section{Summary Questions}

\end{document}
