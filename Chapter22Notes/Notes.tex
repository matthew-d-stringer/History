\documentclass[12pt]{article} %was 12pt
\usepackage{tikz}
\usetikzlibrary{matrix,arrows}
\usepackage{graphicx}
\usepackage{amsmath} 
\usepackage{amssymb} 
\usepackage{amsthm}
\usepackage{enumitem} % to change appearance of enum and item environments
\usepackage{framed}
\usepackage{color}
\usepackage{multicol}
\usepackage [english]{babel}
\usepackage [autostyle, english = american]{csquotes}
\MakeOuterQuote{"}
%Matthew
%\usepackage[utf8]{inputenc}
%\usepackage[english]{babel}
%end Matthew

%\usepackage{yhmath}
% adjust the margins using the geometry package
\usepackage[left=0.75in, right=0.75in, top=0.75in, bottom=0.75in]{geometry}
\usepackage[parfill]{parskip}
 %\usepackage{mathpazo}
%\usepackage{euler}


% customize the headers using the fancyhdr package
\usepackage{fancyhdr}
\pagestyle{fancy}

\usepackage{hyperref}

\usepackage{mathrsfs}
\usepackage{mathtools}

\renewcommand{\headrulewidth}{0.4pt}
\renewcommand{\footrulewidth}{0.4pt}

\newenvironment{hwproblem}[1]{{\large \bfseries Problem #1\,:} \begin{trivlist}\item[]\vspace{-0.5ex}}{\end{trivlist}\vspace{3ex}}
\newenvironment{hwproblemquestion}[1]{{\bfseries Question #1\,:} \begin{trivlist}\item[]\vspace{-1.5ex}}{\end{trivlist}\vspace{3ex}}


% customize enumerate and itemize environments
\setlist[itemize]{labelsep=1ex,itemsep=1.5ex,parsep=0ex,leftmargin=4ex,topsep=0.5ex}
\setlist[enumerate]{labelsep=1ex,itemsep=1.5ex,parsep=0ex,leftmargin=4ex,topsep=0.5ex}




\rhead{Honors US History}
\lhead{Matthew Stringer} % your name
\lfoot{}
\cfoot{Notes}  % change to the corresponding number
\rfoot{\thepage}

\setlength{\headheight}{15pt} 
\title{Chapter 22 Notes} % change to the corresponding number
%\date{September 5, 2018}
\author{Matthew Stringer} % your name and ID number

% Anything above the \begin{document} is the template. If you wish to start a new document using this template, erase everything inside of the \begin{document}...\end{document}
\allowdisplaybreaks
\setlength{\parindent}{4em}
\begin{document}
\maketitle
\tableofcontents
\newpage

\section{Politics (Pages 475-477)}
\begin{enumerate}
	\item The Republican party dominated both the White House and Congress.
	\item Republicans did not want laissez-faire economics but rather wanted limited government regulation as an aid to stabilizing business.
		The regulatory commissions established in the Progressive era were now ran by appointees who were more sympathetic to business than
		the general public. The main idea was that the nation would benefit if business and the pursuit of profits took the lead in developing
		the economy.
	\item 
	\begin{itemize}
		\item Warren Harding
		\begin{itemize}
			\item Appointed able men to his cabinet including 
			\begin{itemize}
				\item Presidential Candidate and Supreme Court justice Charles Evans Hughes to be secretary of state
				\item An admired former mining engineer and Food Administration leader Herbert Hoover to be \newline secretary of commerce
				\item Pittsburgh industrialist and millionaire Andrew Mellon to be secretary of the treasury
			\end{itemize}
			\item Domestic Policy
			\begin{itemize}
				\item Reduced income tax
				\item Increased tariff rates through Fordney-McCumber Tariff Act of 1922
				\item established the Bureau of the Budget.
				\item Pardoned Eugene Debs
			\end{itemize}
			\item Scandals and Death
			\begin{itemize}
				\item Appointed incompetent and dishonest men to fill imported position including
				\begin{itemize}
					\item Secretary of Interior Albert B. Fall (Accepted bribes for granting oi leases near Teapot Dome, Wyoming)
					\item Attorney General Harry M. Daugherty (Accepted bribes for agreeing not to prosecute certain criminal suspects)
				\end{itemize}
			\end{itemize}
		\end{itemize}
		\item Calvin Coolidge
		\begin{itemize}
			\item The Election of 1924
			\begin{itemize}
				\item Calvin Coolidge because the Democrats voted for a 3rd party candidate
				\item Surprisingly, the 3rd party did very well
			\end{itemize}
			\item Vetoes and Inaction
			\begin{itemize}
				\item Calvin Coolidge did little but except keep a close watch on the budget.
				\item Coolidge vetoed even the acts of the Republican majority.
				\item This included bonuses for World War 1 veterans.
				\item This also included an act (the McNary-Haugen Bill of 1928) to help 
					farmers as crop prices fell.
			\end{itemize}
		\end{itemize}
	\end{itemize}
\end{enumerate}

\section{Economics (Pages 477-479)}
\begin{enumerate}
	\item Strengths and Weaknesses of 1920's economy:
	\begin{itemize}
		\item Strengths
		\begin{itemize}
			\item Unemployment was below 4\%
			\item The standard of living for Americans improved significantly
			\item Indoor plumbing and central heating became commonplace
			\item Two-thirds of all homes had electricity
		\end{itemize}
		\item Weaknesses
		\begin{itemize}
			\item Prosperity was not universal
			\item 40\% of families were in the poverty range, struggling to live off of \$1,500 a year
			\item Farmers did not participate in the booming economy
		\end{itemize}
	\end{itemize}
	\item The most significant cause of business prosperity was increased productivity because it 
		allowed the mass production of many new consumer goods of the 1920's. Without the assembly line,
		many new goods would take too long to manufacture in order to make a profit.
	\item The automobile changed life in 
	\begin{itemize}
		\item Social/Cultural changes
		\begin{itemize}
			\item Affected: 
			\begin{itemize}
				\item Traveling for pleasure
				\item Commuting to work
				\item dating
			\end{itemize}
			\item Created new problems including:
			\begin{itemize}
				\item Traffic jams
				\item Injuries and deaths due to accidents
			\end{itemize}
		\end{itemize}
		\item Economic Impact
		\begin{itemize}
			\item Other industries such as steel, glass, rubber, gasoline, and highway construction
				depended on automobile sales.
			\item Replaced the railroad industry as the key promoter of economic growth.
		\end{itemize}
		\item Political Effects
		\begin{itemize}
			\item In 1929, over 26.5 million automobiles were registered
			\item There was an average of one car per family
		\end{itemize}
	\end{itemize}
\end{enumerate}

\section{Culture and Society (Pages 479-486)}
\begin{enumerate}
	\item This era was in so much cultural change because there was a huge growth in cities. The culture
		of cities was based on popular tastes, morals, and habits of mass transportation.
\end{enumerate}
\subsection{Entertainment and Arts}
\begin{enumerate}[resume]
	\item There were many new forms of entertainment in the 1920's including 
	\begin{itemize}
		\item Jazz music
		\item The radio
		\item Hollywood movies (sound movies arrived in 1927 and also improved the popularity of movies)
	\end{itemize}
	\item Technology allowed people from all across the country to enjoy the same entertainment. This 
		created many public figures such as Jack Dempsey(a boxer), Gertrude Ederle(a swimmer), Jim Thorpe(a football player),
		and Babe Ruth(a baseball player).
	\item Writers like F. Scott Fitzgerald, Ernest Hemingway, Sinclair Lewis, Ezra Pound, T. S. Eliot, 
		and Eugene O'Neil wrote about the disillusionment of ideals and the materialism of business-oriented
		culture.
	\item "Art Deco" has repeating patterns that consist of geometric shapes that are outlined in gold on
		a blue or black background.
\end{enumerate}
\subsection{Gender}
\begin{enumerate}[resume]
	\item Women in the 1920's did not vote in a bloc, and instead adopted the party preferences of their
		husbands or fathers. This caused little change to be made because women did not vote for candidates
		that had their beliefs or granted them more rights.
	\item Women worked in cities as clerks, nurses, teachers, and domestics, but received lower wages than
		men. Women would dress with the flapper look and took office jobs until they married, when they were
		expected to quit their jobs and settle down as a woman. Housework did not change, even though it became
		easier with new consumer goods.
\end{enumerate}
\subsection{Harlem Renaissance}
\begin{enumerate}[resume]
	\item The leading figures of the Harlem Renaissance were Countee Cullen, Langston Hughes, 
		James Weldon Johnson, and Claude McKay.
	\item The Harlem Renaissance allowed African Americans to speak out about their African American heritage.
	\item Marcus Garvey advocated individual and racial pride for African Americans and developed Black nationalism.
		I think many African American leaders disagreed with his goals because he wanted to further separate 
		African Americans from Whites.
\end{enumerate}

\section{Values in Conflict}
\begin{enumerate}
	\item People were divided between 
	\begin{itemize}
		\item young and old 
		\item urban modernist and rural fundamentalist
		\item prohibitionist and antiprohibitionist
		\item nativist and foreign-born
	\end{itemize}
	\item Issues that fueled these division were 
	\begin{itemize}
		\item Religion (Modernism, Fundamentalism, or Revivalist)
		\item Prohibition
		\item Nativism 
	\end{itemize}
	\item Nativism is the policy of protecting the interests of native-born or established inhabitants against those
		of immigrants. 
	\item It caused the United States to set quotas on how many immigrants could come from specific 
		nations. For example it was used to reduce the number of immigrants from southern and eastern Europe.
\end{enumerate}

\section{Ku Klux Klan}
\begin{enumerate}
	\item The Klan attracted new members because of the popular silent film, \textit{Birth of a Nation}. It also used
		modern advertising techniques to grow to 5 million members by 1925.
	\item They were hostile to Blacks, Catholics, Jews, foreigners, and suspected Communists. They dressed in white 
		hoods, burned crosses, and applied vigilante justice, using whips, tar and feathers, and the hangman's noose.
		They also had a strong political influence. In Indiana and Texas, its support was necessary to winning elections
		to state and local offices.
	\item My image of Klan didn't have the same level of political influence as the real one did. I didn't imagine that
		the Klan's support would be a requirement to getting elected in several states. I also didn't realize that the 
		Klan targeted Catholics, Jews, foreigners, and Communists.
\end{enumerate}

\section{Summary Questions}
\begin{enumerate}
	\item The 1920's was a decade of change for many reasons. First, the invention of the automobile drastically changed
		the way people dressed and travelled in the city. It also transformed our cities and the effects are still noticeable
		today. Second, there was the rise of Jazz, radio, and movies. These revolutionized the entertainment industry by allowing
		people from across the country to consume the same media. Finally, expectations for women changed dramatically. Unlike
		before, women had jobs and were allowed to divorce their husbands.
	\item The quote applies to the 1920's because politically, the nation changed little. During the presidency of Calvin
		Coolidge, the government focused on keeping an eye on the budget and little else. 
	\item What was the reasoning behind the few actions of Calvin Coolidge? Why didn't any focus on the farmers' economic struggle.
\end{enumerate}

\end{document}
