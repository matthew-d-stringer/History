\documentclass[12pt]{article}
\usepackage{HistoryTemplate}
\usepackage{tcolorbox}
\usepackage{microtype}

\title{Chapter 24 Notes}
\date{}
\author{Matthew Stringer}

\lhead{Chapter 24 Notes}
\rhead{Matthew Stringer}
\cfoot{Honors History}
\rfoot{\thepage}

\newcounter{questioncnt}[section]
\newenvironment{question}{
    \refstepcounter{questioncnt}
    % \begin{figure}
    \begin{minipage}{\textwidth}
    \textbf{Question \#\thequestioncnt}
    \newline
    % \begin{tcolorbox}[width=5in]
    \begin{tcolorbox}[width=\linewidth]
}{
    \end{tcolorbox}
    \end{minipage}
    % \end{figure}
}

\newenvironment{answer}{
    \newline
    \textbf{Answer}
    \newline
}{
}

\begin{document}
    \maketitle
    \begin{question}
        In the introduction paragraphs, it states: 
        \emph{``...Americans were gripped by fear for their survival.. This 
        depression of the 1930s felt different"} \\
        Why were Americans afraid? Summarize the reasons given.
    \end{question}

    \begin{question}
        What happened to the stock market in October 1929?
    \end{question}
    \begin{answer}
        In October 1929, the stock market crashed twice. First, on Black Thursday, 
        and later on Black Tuesday. On Black Thursday, the was excessive selling
        in Wall Street causing stock prices to plummet.
    \end{answer}

    \begin{question}
        Summarize the economic collapse:
    \end{question}
    
    \begin{question}
        Did government policies help or hurt the economic stability? Why/how?
    \end{question}
    \begin{answer}
        Government policies hurt economic stability significantly. First, the 
        government did little to control and regulate business. Congress also
        created tariffs that promoted industry but also hurt farmers. Finally,
        the Federal Reserve failed to stabilize banks, the money supply, and
        prices. Because of this, people panicked and tried to get their money
        out of banks, which created more bank failures.
    \end{answer}

    \begin{question}
        Describe President Hoover's philosophy to responding to the economic crisis.
        What did he do and why?
    \end{question}
    \begin{answer}
        President Hoover believed that prosperity would quickly return. Hoover 
        believed that businesses should not cut wages and unions not to strike,
        and private charities should increase their efforts. Hoover was afraid 
        that government intervention would destroy the free market's self 
        reliance. Eventually, Hoover realized that there must be government action
        however, he believed that this action should come from the state.
    \end{answer}

    \begin{question}
        What was Hawley-Smoot Tariff? Did it help or hurt?
    \end{question}
    \begin{answer}
        In June 1930, Hoover creased the tariff rates to the highest in history.
        This tariff set tax increased tariffs from 31\% to 49\%. Because of this 
        foreign nations also enacted higher tariffs. This resulted in a large 
        reduction in global trade and continued the depression.
    \end{answer}

    \begin{question}
        What were the major events of 1932 in response to the continued economic crisis?
    \end{question}
    \begin{answer}
        1932 was the worst year of the depression.
        \begin{itemize}
            \item \textbf{Unrest on the Farms}\newline
                Farmers banded together to stop banks from foreclosing on farms
                and evicting people from their homes. Farmers from the Midwest
                formed the Farm Holiday Association, which tried to increase the
                prices of crop prices by stopping all harvested grain from reaching
                the market for a year.
            \item \textbf{Bonus March}\newline
                In summer of 1932, about one thousand unemployed WWI veterans marched
                to Washington D.C to demand bonuses promised to them in 1945. Eventually,
                thousands of more veterans brought their wives and children to camp in 
                shanty towns outside the Capitol. After 2 veterans were killed, Hoover
                ordered his General to break up the encampment. General Douglas MacAuthor
                used tanks and tear gas to destroy the shantytown.
        \end{itemize}
    \end{answer}

    \begin{question}
        List 3-5 details from FDR's biography.
    \end{question}
    \begin{answer}
        \begin{enumerate}
            \item FDR came from a wealthy New York family and admired his cousin 
                Theodore. 
            \item FDR was paralyzed by polio, however he continued to pursue politics
                even though he was wealthy enough to retire.
            \item FDR's wife was the most active first lady in history and wrote newspapers
                gave speeches and travelled the country. Eleanor was the president's
                social conscience and influenced him to support minorities and the less 
                fortunate.
        \end{enumerate}
    \end{answer}

    \begin{question}
        How was FDR's philosophy on responding to the depression different from Hoover's?
    \end{question}

\end{document}