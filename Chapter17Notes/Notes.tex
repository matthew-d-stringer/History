\documentclass[12pt]{article} %was 12pt
\usepackage{tikz}
\usetikzlibrary{matrix,arrows}
\usepackage{graphicx}
\usepackage{amsmath} 
\usepackage{amssymb} 
\usepackage{amsthm}
\usepackage{enumitem} % to change appearance of enum and item environments
\usepackage{framed}
\usepackage{color}
\usepackage{multicol}
%Matthew
%\usepackage[utf8]{inputenc}
%\usepackage[english]{babel}
%end Matthew

%\usepackage{yhmath}
% adjust the margins using the geometry package
\usepackage[left=0.75in, right=0.75in, top=0.75in, bottom=0.75in]{geometry}
\usepackage[parfill]{parskip}
 %\usepackage{mathpazo}
%\usepackage{euler}


% customize the headers using the fancyhdr package
\usepackage{fancyhdr}
\pagestyle{fancy}

\usepackage{hyperref}

\usepackage{mathrsfs}
\usepackage{mathtools}

\renewcommand{\headrulewidth}{0.4pt}
\renewcommand{\footrulewidth}{0.4pt}

\newenvironment{hwproblem}[1]{{\large \bfseries Problem #1\,:} \begin{trivlist}\item[]\vspace{-0.5ex}}{\end{trivlist}\vspace{3ex}}
\newenvironment{hwproblemquestion}[1]{{\bfseries Question #1\,:} \begin{trivlist}\item[]\vspace{-1.5ex}}{\end{trivlist}\vspace{3ex}}


% customize enumerate and itemize environments
\setlist[itemize]{labelsep=1ex,itemsep=1.5ex,parsep=0ex,leftmargin=4ex,topsep=0.5ex}
\setlist[enumerate]{labelsep=1ex,itemsep=1.5ex,parsep=0ex,leftmargin=4ex,topsep=0.5ex}




\rhead{Honors US History}
\lhead{Matthew Stringer} % your name
\lfoot{}
\cfoot{Notes}  % change to the corresponding number
\rfoot{\thepage}

\setlength{\headheight}{15pt} 
\title{Chapter 17 Notes} % change to the corresponding number
%\date{September 5, 2018}
\author{Matthew Stringer} % your name and ID number

% Anything above the \begin{document} is the template. If you wish to start a new document using this template, erase everything inside of the \begin{document}...\end{document}
\allowdisplaybreaks
\setlength{\parindent}{4em}
\begin{document}
\maketitle
\tableofcontents
\newpage

\section{The West: Settlement of the Last Frontier}
\begin{enumerate}
	\item After the Civil War, Americans began settling in the West including
	\begin{enumerate}
		\item the Great Plains
		\item the Rocky Mountains
		\item the Western Plateau
	\end{enumerate}
	\item Before 1860, lands west of the Mississippi River were considered "the Great American Desert"
	\item There were 250,000 American Indians living in the West in 1865
	\item Taking over the west nearly exterminated the buffalo and seriously damaged the environment
\end{enumerate}

\subsection{The Cattle Frontier}
\begin{enumerate}
	\item Ranchers realized that they could continue the traditions of the cattle business in the 
		late 1800's from the Mexicans
	\item In the 1860's there were about 5 million head of cattle that roamed freely over the Texas
		grasslands
	\item The Texas cattle business was easy to get into because both the cattle and the grass were
		free
	\item Cattle drives began to end in 1880's because overgrazing destroyed grass and a winter 
		blizzard and drought of 1885-1886 killed 90\% of cattle
	\item Also, homesteaders used barbed wire fencing to cut off access to the formerly open range
	\item The wild west was largely tamed by the 1890's
\end{enumerate}

\subsection{The Farming Frontier}
\begin{enumerate}
	\item The Homestead Act offered 160 acres of public land free to any family that settled it for
		5 years
	\item 500,000 families took advantage of this act but 5 times as many families had to purchase
		their land
	\item Homesteaders discovered that 160 acres was not adequate for farming the Great Plains due to
		\begin{enumerate}
			\item Long spells of severe weather
			\item falling prices for their crops
			\item cost of new machinery
		\end{enumerate}
	\item This caused the failure of two-thirds of the homesteader's farms
	\item Those who survived used deep-plowing to use the moisture available and learned to plant 
		hardy strains of Russian wheat that withstood extreme weather
	\item Also, dams and irrigation saved many western farmers 
\end{enumerate}

\section{American Indians in the West}
\begin{enumerate}
	\item In 1865, there were dozens of different cultural and tribal groups of American Indians.
	\item Two-thirds of the western tribal groups lived in the Great Plains
\end{enumerate}

\subsection{Reservation Policy}
\begin{enumerate}
	\item In the 1830's, President Andrew Jackson's policy of removing eastern American Indians to
		the West was based on the belief that land west of the Mississippi would always remain 
		"Indian country"
	\item In 1851, the federal government began to assign the Plains tribes large tracts of lands
		known as reservations - with definite boundaries
\end{enumerate}

\subsection{Indian Wars}
\begin{enumerate}
	\item In the 19th century, the settlement of thousands of miners, ranchers, and homesteaders on
		American Indian lands lead to violence
	\item Fighting between U.S. troops and Plains Indians was often brutal and the Army was 
		responsible for several massacres
	\item In 1866, Captain William Fetterman was wiped out by Sioux warriors
	\item After these wars, there were more treaties that attempted to isolate Plains Indians to 
		smaller reservations
	\item Gold miners refused to stay off American Indians' lands 
	\item The Indian Appropriation Act of 1871 ended recognition of tribes as independent nations
		by the federal government
	\item The constant pressure of the U.S. Army forced tribe after tribe to comply with 
		Washington's terms
	\item The slaughter of most of the buffalo by the early 1880's doomed the way of life of the 
		Plains people
	\item In the December of 1890, the U.S. Army gunned down more than 200 American Indian men, women,
		and children in the "battle" (massacre) of Wounded Knee in the Dakotas
\end{enumerate}

\subsection{Assimilationists}
\begin{enumerate}
	\item The book \textit{A Century of Dishonor} by Helen Hunt Jackson created sympathy for American
		Indians, however, it also generated support for ending Indian culture through assimilation
	\item Reformers advocated formal education, job training, and conversion to Christianity
	\item Boarding schools taught American Indian children white culture, farming, and industrial skills
\end{enumerate}

\subsection{Dawes Severalty Act (1887)}
\begin{enumerate}
	\item This act was designed to break up tribal organizations which many felt kept American 
		Indians from becoming "civilized" citizens
	\item The act divided tribal lands into plots of up to 160 acres, depending on family size, U.S.
		citizenship was granted to those who stayed on the land for 25 years
	\item The government distributed 47 million acres of land to American Indians, however, 90 
		million acres of former reservation land, often the best land, was sold over the years to white
		settlers by the government, speculators, or American Indians
	\item The new policy failed and by the turn of the century, disease and poverty had reduced the
		American Indian population to just 200,000 persons
\end{enumerate}
\end{document}
