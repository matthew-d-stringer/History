\documentclass[12pt]{article} %was 12pt
\usepackage{tikz}
\usetikzlibrary{matrix,arrows}
\usepackage{graphicx}
\usepackage{amsmath} 
\usepackage{amssymb} 
\usepackage{amsthm}
\usepackage{enumitem} % to change appearance of enum and item environments
\usepackage{framed}
\usepackage{color}
\usepackage{multicol}
%Matthew
%\usepackage[utf8]{inputenc}
%\usepackage[english]{babel}
%end Matthew

%\usepackage{yhmath}
% adjust the margins using the geometry package
\usepackage[left=0.75in, right=0.75in, top=0.75in, bottom=0.75in]{geometry}
\usepackage[parfill]{parskip}
 %\usepackage{mathpazo}
%\usepackage{euler}


% customize the headers using the fancyhdr package
\usepackage{fancyhdr}
\pagestyle{fancy}

\usepackage{hyperref}

\usepackage{mathrsfs}
\usepackage{mathtools}

\renewcommand{\headrulewidth}{0.4pt}
\renewcommand{\footrulewidth}{0.4pt}

\newenvironment{hwproblem}[1]{{\large \bfseries Problem #1\,:} \begin{trivlist}\item[]\vspace{-0.5ex}}{\end{trivlist}\vspace{3ex}}
\newenvironment{hwproblemquestion}[1]{{\bfseries Question #1\,:} \begin{trivlist}\item[]\vspace{-1.5ex}}{\end{trivlist}\vspace{3ex}}


% customize enumerate and itemize environments
\setlist[itemize]{labelsep=1ex,itemsep=1.5ex,parsep=0ex,leftmargin=4ex,topsep=0.5ex}
\setlist[enumerate]{labelsep=1ex,itemsep=1.5ex,parsep=0ex,leftmargin=4ex,topsep=0.5ex}




\rhead{Honors US History}
\lhead{Matthew Stringer} % your name
\lfoot{}
\cfoot{Notes}  % change to the corresponding number
\rfoot{\thepage}

\setlength{\headheight}{15pt} 
\title{Chapter 15 Notes} % change to the corresponding number
%\date{September 5, 2018}
\author{Matthew Stringer} % your name and ID number

% Anything above the \begin{document} is the template. If you wish to start a new document using this template, erase everything inside of the \begin{document}...\end{document}
\allowdisplaybreaks
\setlength{\parindent}{4em}
\begin{document}
\maketitle
\tableofcontents
\newpage

\section{Presidential Leadership and Reconstruction}
\subsection{Lincoln's Policies (1863-1865)}
\subsubsection{Proclamation of Amnesty and Reconstruction}
\begin{enumerate}
	%\item Lincoln saw the Confederates as a "disloyal minority" not traitors.
	\item Lincoln set up a process for political reconstruction
	\item Lincoln gave presidential pardons to Confederates that:
	\begin{enumerate}
		\item took an oath of allegiance to the Union
		\item accepted the emancipation of slaves.
	\end{enumerate}
	\item A state could be considered legitimate if at least 10\% of the voters took the loyalty oath.
	\item This required states to rewrite their constitution in order to eliminate slavery.
\end{enumerate}

\subsubsection{Wade-Davis Bill}
\begin{enumerate}
	\item Many Republicans objected to Lincoln's 10\% plan
	\item The Wade-Davis Bill required 50\% of voters to take a loyalty oath and only non-Confederates
		to vote for a new constitution.
	\item Lincoln pocket-vetoed the bill.
\end{enumerate}

\subsubsection{Freedmen's Bureau}
\begin{enumerate}
	\item Congress created a new agency, the Freedmen's Bureau
	\item Acted as an early welfare agency, providing food, shelter, and medical aid for those made
		destitute by the Civil War.
	\item Created 3,000 schools for freed blacks including colleges.
	\item Taught 200,000 African Americans how to read.
\end{enumerate}

\subsubsection{Lincoln's Last Speech}
\begin{enumerate}
	\item Lincoln encouraged Northerners to accept Louisiana as a state.
	\item Lincoln's suggested that he had become more progressive 
	\item Hope for lasting reform was devastated after Lincoln's Assassination.
\end{enumerate}

\subsection{Andrew Johnson and Reconstruction (1865-1868)}
\begin{enumerate}
	\item Johnson was the only Confederate from Tennessee that was loyal to the Union
	\item Republicans picked him to Lincoln's running mate in 1864 in order to encourage pro-Union
		to vote for the Republican Party.
	\item Johnson became the wrong person for the job due to his white supremacy and was bound to 
		clash with Republicans in Congress.
\end{enumerate}

\subsubsection{Johnson's Reconstruction Policy}
\begin{enumerate}
	\item Johnson issued his own Reconstruction proclamation that was similar to Lincoln's 10\% plan.
	\item It provided for the disfranchisement (loss of the right to vote and hold office) of 
	\begin{enumerate}
		\item All former leaders and officeholders of the Confederacy
		\item Confederates with more than \$20,000 in taxable property.
	\end{enumerate}
	\item Johnson still retained the power to pardon to "disloyal" Southerners.
	\item This was an escape clause for wealthy planters which was used frequently.
\end{enumerate}

\subsubsection{Southern Governments of 1865}
\begin{enumerate}
	\item After 8 months, 11 of ex-Confederate states qualified under the Reconstruction plan to 
		become parts of the Union.
	\item None of the new constitutions extended voting rights to blacks. 
	\item Former leaders of the Confederacy won seats in Congress.
\end{enumerate}

\subsubsection{Black Codes}
\begin{enumerate}
	\item Southern state legislatures adopted Black Codes that restricted the rights and movements
		of the former slaves.
	\item Codes 
	\begin{enumerate}
		\item prohibited blacks from either renting land or borrowing money to buy land
		\item placed freedmen into a form of semibondage by forcing, as "vagrants" and "apprentices"
			to sign work contracts
		\item prohibited blacks from testify against whites in court.
	\end{enumerate}
	\item The contract-labor system seemed little different from slavery.
	\item Northern Republicans in Congress refused to seat duly elected representatives and senators
		from ex-Confederate states.
\end{enumerate}

\subsubsection{Johnson's Vetoes}
\begin{enumerate}
	\item Johnson vetoed a bill increasing the services and protection offered by Freedmen's Bureau
	\item He also vetoed a bill that nullified the Black Codes and guaranteed full citizenship to 
		African Americans
	\item This alienated even moderate Republicans 
\end{enumerate}

\section{Congressional Reconstruction}
\begin{enumerate}
	\item Angry response by members of Congress led to the second round of Reconstruction.
	\item Featured policies that were
	\begin{enumerate}
		\item hasher on Southern whites
		\item more protective of freed African Americans
	\end{enumerate}
\end{enumerate}

\subsection{Radical Republicans}
\begin{enumerate}
	\item Republicans were divided between
	\begin{enumerate}
		\item moderates (were chiefly concerned with economic gains for white middle class)
		\item radicals (championed civil rights for blacks)
	\end{enumerate}
	\item More Republicans became radical in fear that the Democratic party might become dominant
		again
	\item Since the federal census counted all people equally, the South would have more 
		representatives in Congress
	\item Charles Sumner of Massachusetts was the leading Radical Republican (who got cane by Brooks)
	\item Radical Republicans endorsed several liberal causes including
	\begin{enumerate}
		\item women's suffrage
		\item rights for labor unions
		\item civil rights for Northern African Americans
	\end{enumerate}
\end{enumerate}

\subsubsection{Civil Rights Act of 1866}
\begin{enumerate}
	\item First, congressional Reconstruction worked to override Johnson's vetoes of the Freedmen's
		Bureau Act and the first Civil Rights Act.
	\item This Civil Rights Act stated that all African Americans are U.S. citizens
	\item It also attempted to provide a legal shield against Black Codes
	\item Republicans looked for a permanent solution by adding the Fourteenth Amendment.
	\item The Fourteenth Amendment
	\begin{enumerate}
		\item declared that all persons born or naturalized in the U.S. were citizens
		\item obligated the states to respect the rights of U.S. citizens and provide them with 
			"equal protection of the laws" and "due process of law"(clauses full of meaning for 
			future generations)
		\item disqualified former Confederate political leaders from holding either state or federal
			offices
		\item repudiated the debts of the defeated governments of the Confederacy
		\item penalized a state if it kept any eligible(\textit{what counts as eligible?}) person 
			from voting by reducing that state's proportional representation in Congress and the 
			electoral college
	\end{enumerate}
\end{enumerate}

\subsubsection{The Election of 1866}
\begin{enumerate}
	\item Johnson took to the road in order to attack his opponents
	\item Johnson said that equal rights for blacks would resort in "Africanized" society
	\item Republicans called Johnson a drunkard and a traitor
	\item They also campaigned using the "waving the bloody shirt" where they inflamed the anger of
		Northerners by reminding them of the war
	\item Republicans said that Southerners were Democrats and branded them as a party of rebellion
		and treason.
	\item Republicans won the election and had a majority in the House and Senate.
\end{enumerate}

\subsubsection{Reconstruction Acts of 1867}
\begin{enumerate}
	\item Congress passed 3 Reconstruction acts in 1867 
	\item These put the South under military occupation
	\item They divided the states into 5 military districts, each under control by the Union
	\item These acts also required that a state must ratify the 14th Amendment and place guarantees
		in its constitution for granting the right to vote to all adult males
\end{enumerate}

\subsection{Impeachment of Andrew Johnson}
\begin{enumerate}
	\item Congress passed the Tenure of Office Act which may have been unconstitutional
	\item It prohibited the president from removing a federal official or military commander without
		approval of the Senate
	\item This was meant to protect Radical Republicans in the cabinet, like Secretary of War Edwin
		Stanton
	\item Johnson challenged the new law by dismissing Stanton on his own authority
	\item The House charged him with impeachment, making Johnson the first president to be impeached
	\item After a 3 month trial, Johnson was not removed from office
\end{enumerate}

\subsection{Reforms After Grant's Election}

\subsubsection{The Election of 1868}
\begin{enumerate}
	\item Republicans turned to a war hero General Ulysses S. Grant
	\item Grant had no political experience
	\item He only won 300,000 more popular votes than his opponent 
	\item He was only able to win because 500,000 blacks voted for him
\end{enumerate}

\subsubsection{Fifteenth Amendment}
\begin{enumerate}
	\item Republican majorities in Congress in 1869 created an amendment to protect African Americans
	\item The 15th Amendment prohibited any state from denying or abridging a citizen's right to 
		vote "on account of race, color, or previous condition of servitude"
	\item The amendment was ratified in 1870
\end{enumerate}

\subsubsection{Civil Rights Act of 1875}
\begin{enumerate}
	\item This was the final civil rights reform
	\item It guaranteed equal accommodations in public places (hotels, railroads, and theaters) and
		prohibited courts from excluding African Americans from juries
	\item These were poorly enforced because Republicans were frustrated trying to reform an unwilling
		South
	\item They also feared loosing white votes in the North
\end{enumerate}

\section{Reconstruction in the South}
\begin{enumerate}
	\item Republican party in the South dominated governments in ex-Confederate states
	\item Once Republicans felt that they were satisfied, troops were withdrawn
	\item In Tennessee troops stayed for less than a year
	\item In Florida troops stayed for nine years
\end{enumerate}

\subsection{Composition of the Reconstruction Governments}
\begin{enumerate}
	\item In every Republican state government in the South except one, whites were the majority in
		both houses of legislature
	\item In South Carolina, freedmen controlled the lower house in 1873
\end{enumerate}

\subsubsection{"Scalawags" and "Carpetbaggers"}
\begin{enumerate}
	\item Southern Republicans were called "Scalawags"
	\item Northern newcomers were "Carpetbaggers"
	\item Northerners migrated South after the war because they
	\begin{enumerate}
		\item were interested investors
		\item were ministers and teachers with humanitarian goals
		\item wanted to plunder
	\end{enumerate}
\end{enumerate}

\subsubsection{African American Legislators}
\begin{enumerate}
	\item African Americans who held elective office were educated property holders who were 
		predominantly moderate
	\item Republicans in the South sent two African Americans to the Senate
	\item They also sent a dozen African Americans to the House of Representatives
	\item This caused a bitter resentment among ex-Confederates
\end{enumerate}

\subsection{Evaluating the Republican Record}
\subsubsection{Accomplishments}
\begin{enumerate}
	\item Liberalized state constitutions in the South
	\begin{enumerate}
		\item Universal male suffrage
		\item Property rights for women
		\item Debt relief
		\item Modern penal codes
	\end{enumerate}
	\item Promoted building of
	\begin{enumerate}
		\item roads
		\item railroads
		\item other internal improvements
	\end{enumerate}
	\item Established state institutions such as
	\begin{enumerate}
		\item hospitals
		\item asylums
		\item homes for the disabled
		\item public school systems in the South
	\end{enumerate}
\end{enumerate}
\subsubsection{Failures}
\begin{enumerate}
	\item Depicted as wasteful and corrupt
	\item Republican politicians took advantage of their power to take kickbacks
	\item Corruption occurred throughout the country
	\item No party was immune
\end{enumerate}

\subsection{African Americans Adjusting to Freedom}
\subsubsection{Building Black Communities}
\begin{enumerate}
	\item Ex-slaves viewed emancipation as an opportunity for achieving independence from white 
		control 
	\item African Americans founded hundreds of churches after the war
	\item Ministers emerged as leaders in African American communities
	\item African Americans use scarce resources to establish independent schools for their children
	\item They paid educated African Americans to become their teachers
	\item Black Colleges were established to prepare African American ministers and teachers
\end{enumerate}
\subsubsection{Sharecropping}
\begin{enumerate}
	\item Landowners tried to convince African Americans to sign a document that forced unrestricted
		labor 
	\item This was slavery except by a different name
	\item When African Americans insisted on autonomy, white landowners began sharecropping
	\item With sharecropping, the landlord provided the tools and land in return for usually half of
		the crops
	\item By 1880, no more than 5\% of Southern African
\end{enumerate}

\section{The North During Reconstruction}
\subsection{Greed and Corruption}
\subsubsection{Rise of Spoilsmen}
\begin{enumerate}
	\item In the 1870's, leadership of the Republican party passed from reformers to political
		manipulators
	\item These gave jobs and government favors (spoils) to their supporters
\end{enumerate}
\subsubsection{The Compromise of 1877}
\begin{enumerate}
	\item Leaders of the 2 parties agreed that 
	\begin{enumerate}
		\item Democrats would allow Hayes to become president
		\item He would end federal support for the Republicans in the South
		\item Support the building of a Southern transcontinental railroad
	\end{enumerate}
	\item Hayes promptly withdrew the last of the federal troops protecting African Americans
	\item The Supreme Court struck down many Reconstruction laws that protected blacks from 
		discrimination
	\item Most Southern African Americans and whites in the decades after the Civil War remained 
		poor farmers
\end{enumerate}

\end{document}
